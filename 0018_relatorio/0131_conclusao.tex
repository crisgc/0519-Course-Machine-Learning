\section{Conclusão}

Nesta presente análise, investigamos o impacto de vários parâmetros e 
hiperparâmetros no treinamento e desempenho de redes neurais. Exploramos 
diferentes funções de ativação, números de épocas, taxas de aprendizagem, 
arquiteturas de rede, níveis de ruído nos dados e taxa de regularização. 
Cada 
tarefa forneceu ``insight''s sobre como esses fatores influenciam o 
comportamento e a capacidade preditiva das redes neurais.

\begin{comment}
Na tarefa 1, descobrimos que a função de ativação Gelu nas camadas 
intermediárias resultou em uma melhor capacidade de aprendizado em 
comparação 
com a função sigmóide. A função Relu, embora interessante, apresentou um 
comportamento mais ruidoso, sugerindo a necessidade de uma inicialização 
de 
parâmetros e taxa de aprendizagem cuidadosas.

Na tarefa 2, observamos que um número excessivo de épocas de treinamento 
pode 
não trazer benefícios significativos após um certo ponto. As redes neurais 
convergiram para erros semelhantes após 5000 épocas, e a rede treinada com 
apenas 1000 épocas exibiu underfitting. Esta análise destaca a importância 
de 
monitorar o desempenho em diferentes pontos do treinamento.

A tarefa 3 enfatizou a escolha crucial da taxa de aprendizagem. Uma taxa 
muito 
baixa resultou em convergência lenta, enquanto uma taxa muito alta levou a 
um 
comportamento instável. A taxa de 0,05 pareceu ser um bom equilíbrio, 
proporcionando uma convergência suave e um bom desempenho final.

Na tarefa 4, descobrimos que a complexidade da arquitetura da rede deve 
ser 
considerada em relação à quantidade de dados. Uma rede neural mais simples 
pode 
convergir bem e fornecer resultados comparáveis a redes mais profundas, 
dependendo do problema.

A tarefa 5 revelou que níveis mais altos de ruído nos dados resultaram em 
uma 
convergência mais lenta e instável, mas o modelo adaptou-se e aprendeu a 
lidar 
com o ruído. Esta análise destaca a importância de considerar a qualidade 
dos 
dados e a necessidade de lidar com o ruído durante o treinamento.

Por fim, na tarefa 6, vimos que a regularização L2 pode ajudar a prevenir 
o 
overfitting, mas uma regularização muito forte pode levar a underfitting. 
O 
valor de λ=0.001 pareceu ser um bom equilíbrio, permitindo alguma 
regularização 
sem comprometer o desempenho.
\end{comment}

Esta série de tarefas demonstra a importância da seleção cuidadosa 
de parâmetros e hiperparâmetros no treinamento de redes neurais. A escolha 
adequada de funções de ativação, número de épocas, taxa de aprendizagem, 
arquitetura e regularização é crucial para alcançar um bom desempenho e 
evitar 
problemas como \textit{overfitting} e \textit{underfitting}. A análise 
sistemática desses 
fatores é essencial para otimizar os modelos de aprendizado de máquina e 
garantir sua capacidade de generalização para novos dados.